\documentclass[10pt,sanmathserif]{beamer}
%\documentclass[10pt,handout,sanmathserif]{beamer}

% Goal: no more than 32 frames

\setbeamertemplate{navigation symbols}{}

%\definecolor{UniBlue}{RGB}{83,121,170}
%\setbeamercolor{title}{fg=UniBlue}
%\setbeamercolor{frametitle}{fg=UniBlue}


\usepackage{amssymb,bm}
\usepackage[curve,color,dvips,graph]{xypic}
\usepackage{graphics,epsfig}
%\usepackage[usenames]{color}
%\usepackage{pst-node}
\usepackage{mathrsfs}

% For guillemots, 
% \guillemotleft and \guillemotright or just use >> or <<.
\usepackage[T1]{fontenc}
\frenchspacing

%\documentclass[10pt]{beamer}
%\usepackage{amssymb,amscd,amsmath,graphics,amsthm}
%
%\usetheme{Copenhagen}
%\setbeamercovered{transparent}
%\usefonttheme[onlylarge]{structurebold}
%\setbeamerfont*{frametitle}{size=\normalsize,series=\bfseries}
%\setbeamertemplate{navigation symbols}{}
%\setbeamertemplate{footline}[default]


% Standard packages

\usepackage[english]{babel}
\usepackage[latin1]{inputenc}
\usepackage{times}
\usepackage[T1]{fontenc}
\usepackage{epsfig}
\usepackage{pst-node}
\usepackage{tikz-cd} 
\usepackage[all,cmtip]{xy}

% Setup TikZ

\usepackage{tikz}
\usetikzlibrary{arrows}
\tikzstyle{block}=[draw opacity=0.7,line width=1.4cm]

\newenvironment<>{examplefirst}[1]{%
  \setbeamercolor{block title}{fg=black,bg=purple!40!purple!50}%
  \setbeamercolor{block body}{bg=purple!50!white!10}
  \begin{block}#2{#1}}{\end{block}}
\newenvironment<>{examplesecond}[1]{%
  \setbeamercolor{block title}{fg=white,bg=green!50!black!60}%
  \begin{block}#2{#1}}{\end{block}}
\newenvironment<>{examplethird}[1]{%
  \setbeamercolor{block title}{fg=black,bg=red!30!blue!40}%
  \setbeamercolor{block body}{bg=red!30!blue!20}
  \begin{block}#2{#1}}{\end{block}}
  \newenvironment<>{examplefourth}[1]{%
  \setbeamercolor{block title}{fg=white,bg=cyan}%
  \setbeamercolor{block body}{bg=cyan!50!white!10}
  \begin{block}#2{#1}}{\end{block}}



\title[ ] 
{Uniform Convergence\\ for Sequences of Iterates
}

\institute[ ]
{
Butler University

\medskip

}



\date[ ]
{
\today}


\begin{document}

\begin{frame}
  \titlepage
\end{frame}




\begin{frame}
\frametitle{Iterates}

\begin{examplefirst}{Definition}
\vspace{12pt}

Let $f:X\rightarrow X$; define the {$n^{\mbox{\footnotesize{th}}}$ iterate} of $f$ as 
\[f^n := \overbrace{f \circ f\circ...\circ f}^{n}\]

\end{examplefirst}

\emph{Example:}
\begin{center} 
If $f(z)=z^2$, then $f^3(z)=z^2\circ z^2\circ z^2= (((z^2)^2)^2=z^8$
\end{center}

\pause

\begin{examplefirst}{Definition}
We say $x\in X$ is a fixed point of $f\colon X\rightarrow X$ if $f(x)=x$.
\end{examplefirst}

\emph{Example:}
\begin{center} 
If $f(z)=z^2$ then $f(0)=0$. Therefore 0 is a fixed point.  
\end{center}
\end{frame}





\begin{frame}
\frametitle{Old Theorems}

\begin{examplefourth}{Denjoy-Wolff Theorem}

If $\varphi\colon\mathbb D\rightarrow\mathbb D$ is analytic but neither the identity nor a rotation, then there exists a unique point $a\in\overline{\mathbb D}$ so that $\varphi^n$ converges to the constant function $a$ uniformly on compact subsets of $\mathbb D$.

\end{examplefourth}

\pause

We call $a$ the Denjoy-Wolff point:
\begin{itemize}
\item If $a\in\mathbb D$, then $\varphi(a)=a$, and $f$ has no other fixed points;
\item If $a\in\partial\overline{\mathbb D}$, then $\displaystyle \lim_{r\rightarrow1^-} \varphi(ra)=a$.
\end{itemize}

\end{frame}




\begin{frame}{New Theorems}

\begin{examplesecond}{Theorem 1 (Cowen, Ko, Thompson, Tiang 2014)}
Suppose $\varphi\colon\overline{\mathbb D}\rightarrow\mathbb D$ is analytic on $\mathbb D$ and continuous on the boundary, $\partial\mathbb D$. If the Denjoy-Wolff point $a\in\mathbb D$, then $\varphi^n \rightarrow a$ uniformly \emph{on all of $\mathbb D$} if and only if there is $N>0$ such that $\varphi^N(\overline{\mathbb D})\subseteq \mathbb D$.
\end{examplesecond}

\vspace{12pt}
\pause

\begin{examplesecond}{Theorem 2 (Cowen, Ko, Thompson, Tiang 2014)}
Suppose $\varphi\colon\overline{\mathbb D}\rightarrow\mathbb D$ is analytic on $\mathbb D$ and continuous on the boundary, $\partial\mathbb D$, and has Denjoy-Wolff point $a$ with $|a|=1$ and $\varphi'(a)<1$. If $\varphi^N(\mathbb D)\subseteq \mathbb D \cup \{a\}$ for some $N > 0$, then $\varphi^n\rightarrow a$ uniformly on all of $\mathbb D$.
\end{examplesecond}

\vspace{12pt}
\pause

\begin{examplefirst}{
Definition}
We say UCI (\emph{uniformly convergent iterates}) holds for $f\colon X\rightarrow X$ if $f^n$ \textit{converges uniformly on all of $X$ to a constant $a$}.   
\end{examplefirst}
\end{frame}

\begin{frame}%{New Theorems}
\begin{examplethird}{An Example}
$\varphi(z)=\frac{1}{2}z+\frac{1}{2}$ has fixed point at 1 with $\varphi'(1)<1$.  Has UCI.
\end{examplethird}

\vspace{12pt}
\pause

\begin{examplethird}{Another Example}
$\varphi(z)=1/(2-z)$ has fixed point at 1 with $\varphi'(1)=1$.  Has UCI.
\end{examplethird}

\vspace{12pt}
\pause

\begin{examplethird}{Generalized Example}
If $\varphi$ is a linear fractional map with  DW point $|a|=1$ and $\varphi'(a)=1$ but not an automorphism, then UCI.
\end{examplethird}

\vspace{12pt}
\pause

\begin{examplethird}{An Annoying Example}
$\varphi(z)=\frac{(z+1)k+(z-1)}{(z+1)k-(z-1)}$ for $k>1$ has $\varphi(1)=1$ and $\varphi(-1)=-1$.  NO UCI.
\end{examplethird}
\end{frame}



\begin{frame}{More New Theorems}
\begin{examplesecond}{Theorem 3 (Cowen, Ko, Thompson, Tiang 2014)}
If 
\begin{itemize}
\item $\varphi\colon\overline{\mathbb D}\rightarrow\mathbb D$ is analytic on $\mathbb D$,
\item $\varphi$ is continuous on $\partial\mathbb D$, and
\item $\varphi^n(z)\rightarrow a$ uniformly on $\mathbb D$ and $|a|=1$, 
\end{itemize}
then $a$ is the only fixed point of $\varphi$ on the closed disk.
\end{examplesecond}

\vspace{12pt}

\begin{examplesecond}{Theorem 4 (Glickfield, Kaschner 2018)}
If 
\begin{itemize}
\item $\varphi\colon\overline{\mathbb D}\rightarrow\mathbb D$ is analytic on $\mathbb D$ and
\item $\varphi^n(z)\rightarrow a$ uniformly on $\mathbb D$ and $|a|=1$, 
\end{itemize}
then $a$ is the only fixed point of $\varphi$ on the closed unit disk.
\end{examplesecond}
\end{frame}



\begin{frame}{Problem(s)}

\begin{center}
When does UCI hold?\\\pause
When is the fixed point unique?
\end{center}

\vspace{18pt}
\pause

What about $\varphi(z)=\frac{1}{2}z^2+\frac{1}{2}$?  

\vspace{18pt}
\pause

Let's explore different domains:
\begin{itemize}
\item $\mathbb D^2$ (bi-disk)
\item $\{\vec z\in\mathbb C^2\colon\|\vec z\|<1\}$ (unit ball)\pause
\item bi-complex bi-disk and unit ball\pause
\item $(-1,1)$
\item $(-1,1)^2$ (real bi-disk or unit square)
\item $\{\vec z\in\mathbb C^2\colon\|\vec z\|<1\}$ (real unit ball)
\end{itemize}
\end{frame}







\end{document}